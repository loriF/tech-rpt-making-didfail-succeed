\chapter{Introduction}\label{ch:intro}
This should be the first chapter of the technical document if it is chosen to be included. Be sure to uncomment the include statement in the main file in order to publish the contents within this file.

\begin{figure}[htbp]
	\centering
	\includegraphics{assets/abs_image}
	\caption{Sample Figure Caption}
	\label{samplefigure}
\end{figure}

As you can see in Figure \ref{samplefigure}, this is a brief example of how you include graphics in a report with a caption. You can also add optional parameters to the code, for example $\\includegraphics[width=3in]{imagename}$ will create a 3 inch wide image. Similarly $\\includegraphics[height=3in]{imagename}$ will create a 3 inch high image. Parameters can also be combined if separated with a comma.

\begin{equation}
	\begin{array}{l}
		\displaystyle \int 1 = x + C\\
		\displaystyle \int x = \frac{x^2}{2} + C \\
		\displaystyle \int x^2 = \frac{x^3}{3} + C
	\end{array} 
	\label{eq:xdef}
\end{equation}

Also note that Equation \ref{eq:xdef} above shows how to insert equations using math mode. By default these are centered and will be located on their own (not inline with text). If you want to insert equations like $x = \sin \alpha = \cos \beta = \cos(\pi-\alpha) = \sin(\pi-\beta)$, you can invoke math mode in a different manner.

Even though the body of this document contains instructions for Microsoft Word, the information is still useful. Although none of the Word features are available using the methods described in this paper, the resulting \LaTeX\space code is complete enough to show most any common task.

Furthermore, there are many other optional features which can be used by uncommenting the code in the main program if they are disabled. These include but are not limited to $Draft$ mode which places ``DRAFT PENDING RRO APPROVAL'' in the header of the appropriate pages, Acronym Lists, a Glossary, and Index.

\textbf{Using the Glossary and Acronym List}\\
This feature requires Perl to be installed in order to run successfully. Due to this, the Glossary and Acronym List could not be integrated into this document easily.

\textbf{Using the Index}\\
If we want to include references to something like the Glossary\index{Glossary} or the Index\index{Index}, or even using \LaTeX\index{Using \LaTeX}, we can create $\textbackslash index\{Name That Appears in Index\}$ entries that will populate the report index. To compile the index follow the steps below:
\begin{enumerate}
\item Ensure that your \LaTeX\space compiler is configured so MakeIndex runs with the following command... $makeindex.exe -s\mbox{ }\%.ist\mbox{ }\%.idx$
\item Compile the main document with PDFLaTeX
\item Run makeindex.exe from your \LaTeX\space editor
\item Compile the bibliography with BibTeX if you have a bibliography
\item Compile the main document with PDFLaTeX
\item Compile the main document with PDFLaTeX
\item View the output PDF and make sure everything looks appropriate
\end{enumerate}