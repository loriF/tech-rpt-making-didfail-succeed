\chapter{Do Not Use These Word Features}\label{ch:donotuse}

\section{Understand Effects of Using New or Enhanced Features in Word 2007}
Word 2003 users may not be able to change some items that were created by using Word 2007-specific new or enhanced features. For example, if you use a previous version of Word to open a document that contains equations that were created with Word 2007, in Word 2003, the equations will become images that cannot be edited. See Section 2 for more information.

\section{Do Not Delete Section Breaks}
In Word, section breaks control general layout and footers and headers. If you delete a section break, you could merge two sections and unintentionally alter other areas in the document.

If you must delete a section break, inspect the document closely before moving on to the next task and pay particular attention to footers, chapter pagination, and placement of the ``signatures'' (logos) on the cover and title pages.

If you delete a section break by mistake, use the \textit{Undo} button to restore it.

\section{Do Not Add New Styles or Modify Existing Ones}
Many of the styles in this template are interconnected. Altering one style might affect the appear-ance of another one.

Please contact the templates team (\href{mailto:templates@sei.cmu.edu}{templates@sei.cmu.edu}) if you need to modify existing Styles or add new ones.

\section{Do Not Use Fast Saves}
This option saves only the changes to a document instead of saving the entire document. While this feature seems desirable, we recommend not using it; fast saves can cause problems, especially when you are working over a network.

\section{Do Not Use Autocaptions}
Do not use the AutoCaption feature when creating or maintaining your report.

\section{Do Not Delete the Lists}
The Table of Contents, List of Figures, and the List of Tables are automatically generated providing that you use the appropriate style names for each figure and table caption in your document.\footnote{The Word style to use is \textit{Caption}.}

If there is a chance that you will use these lists in your report, do not delete them. Once you've entered some figures and tables in your report, follow these steps to update the list of figures and list of tables:
\begin{enumerate}
\item Click anywhere in the document.
\item Type \textit{CTRL-a}.
\item Right-click and select \textit{Update Field}.
\item Click \textit{Update Entire Table}.
\item Click \textit{OK}.\\The new list is generated and displayed. 
\end{enumerate}
