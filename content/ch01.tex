\chapter{Use These Word Features}\label{ch:usethese}
This section contains hints for creating and maintaining Word files and suggestions for avoiding common mistakes.

\section{Use Styles}
Use Word's ``\index{Styles}styles'' to format everything in your document. You should be able to format every element of your report without having to modify a style, change the point size of a paragraph, change the \index{Fonts}font type of a phrase, or add a style. However, feel free to use the Bold and Italics buttons to format individual words and phrases.

If you find that the SEI report templates lack certain styles, please send email to \\\href{mailto:templates@sei.cmu.edu}{templates@sei.cmu.edu}.

\section{Use Keyboard Shortcuts}
This version of the report template includes the following keyboard shortcuts for applying some of the commonly used Word styles and performing some common Word functions. Table 1 shows only the most common keyboard shortcuts. To view a complete list of Word shortcuts, read the Word MVP FAQ about shortcuts\\
(\href{http://word.mvps.org/FAQs/General/InsertSpecChars.htm}{http://word.mvps.org/FAQs/General/InsertSpecChars.htm}).

% Alternating row colors
\rowcolors{2}{gray!5}{gray!20}

\begin{center}
\begin{longtable}{ll}
\caption{Common Keyboard Shortcuts}{} \label{table1} \\
\textbf{This Keyboard Shortcut} & \textbf{Applies this Word Style} \\
CTRL-Alt-1 & Heading 1 \\
CTRL-Alt-2 & Heading 2 \\
CTRL-Alt-3 & Heading 3 \\
F5 & Body \\
F6 & List Bullet 1 \\
CTRL-F6 & List Bullet 2 \\
F7 & List Numbered 1 \\
CTRL-F7 & List Numbered 2 \\
F8 & List Multilist 1 \\
CTRL-F8 & List Multilist 2 \\
This Keyboard Shortcut & Performs this Action \\
Home & moves to the beginning of the line \\
Shift-Home & moves to the beginning of the document \\
CTRL-$\uparrow$ & moves to the previous paragraph \\
CTRL Shift $\uparrow$ & highlights from the cursor to the beginning of the paragraph \\
CTRL-$\downarrow$ & moves to the next paragraph \\
CTRL Shift $\downarrow$ & highlights from the cursor to the end of the paragraph \\
End & moves to the end of the line \\
Shift-End & moves to the end of the document \\
CTRL $\rightarrow$ & moves to the next word \\
CTRL Shift $\rightarrow$ & highlights the next word \\
CTRL $\leftarrow$ & moves to the previous word \\
CTRL Shift $\leftarrow$ & highlights the previous word \\
CTRL+L & applies left alignment \\
CTRL+E & applies centered alignment \\
CTRL+R & applies right alignment \\
CTRL + J & applies justified alignment \\
CTRL+ Backspace & deletes previous word \\
CTRL + Delete & deletes next word \\
\textbf{This Keyboard Shortcut} & \textbf{Inserts this Character} \\
CTRL Alt C & \textcopyright \\
CTRL Alt R & \textregistered \\
CTRL Alt NumPad - & -- \\
CTRL \raisebox{-1ex}{\textasciitilde} a & \`{a} \\
CTRL Shift \raisebox{-1ex}{\textasciitilde} a & \~{A} \\
CTRL + \raisebox{-1ex}{\textasciitilde}, the letter & \`{a}, \`{e}, \`{i}, \`{o}, \`{u}, \`{A}, \`{E}, \`{I}, \`{O}, \`{U} \\
CTRL + ', the letter & \'{a}, \'{e}, \'{i}, \'{o}, \'{u}, \'{y}, \'{A}, \'{E}, \'{I}, \'{O}, \'{U}, \'{Y} \\
CTRL + Shift + \textasciicircum  , the letter & \^{a}, \^{e}, \^{i}, \^{o}, \^{u}, \^{A}, \^{E}, \^{I}, \^{O}, \^{U} \\
CTRL + Shift + \raisebox{-1ex}{\textasciitilde}, the letter & \~{a}, \~{n}, \~{o}, \~{A}, \~{N}, \~{O} \\
CTRL + Shift + :, the letter & \"{a}, \"{e}, \"{i}, \"{o}, \"{u}, \"{A}, \"{E}, \"{I}, \"{O}, \"{U}
\end{longtable}
\end{center}

\section{Update the Lists}
To update the \index{Lists}lists (e.g., \index{Table of Contents (Usage)}Table of Contents, \index{List of Figures (Usage)}List of Figures, and the \index{List of Tables (Usage)}List of Tables), type \textit{CTRL-a} (hold down the \textit{Control key} and then click the \textit{a} key), right click, and then select \textit{Update Field}.

\section{Understand More About Paragraph Marks}
The mark at the end of each paragraph holds the formatting definitions for that paragraph. When you delete a paragraph mark, you delete the formatting for that paragraph. Keep this in mind when you are editing a document.

Also, the final paragraph mark in a document contains a lot of information about the document. Often when Word files become corrupt, you can recover the file by copying the contents of the document EXCEPT for the final paragraph mark and pasting those contents into a new SEI template.

\section{Use Cross References}
When referring to figures, tables, or headings in the text of your report, use Word's cross reference feature. This practice will help you avoid the need to change these citations every time you move, add, or remove elements that change the name or number of figures, tables, and headings.